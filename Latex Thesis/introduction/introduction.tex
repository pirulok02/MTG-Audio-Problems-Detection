\normallinespacing

\chapter{Introduction}

\bf Audio Problems \rm refer to unwanted or undesired sound contained in an audio file. Many particular issues are considered undesirable, either for mixing purposes or for perceptual quality purposes. Quality of audio recordings can both be subjectively evaluated by percetually detecting audio issues or can be also evaluatted objectively by algorithms detecting patterns or specific features extracted from audio files. This thesis will be focused in the development and evaluation of the objective way to detect audio issues. 

\section{Objectives}
The thesis will have three main objectives:
\begin{itemize}
\item The first is to evaluate and tune existing algorithms, which are included in the Essentia library. 
Designed for detecting defects in songs, which its size is much greater than the target audio files of this thesis. 
\item The second one is to create and develop algorithms to detect other issues common in sound files. Three more issues will 
be evaluated in this thesis, low SNR, bit depth discrepancies and sample rate discrepancies. \\ Low SNR (Signal to Noise Ratio) 
signals are defined by the SNR equation, which quantifies the relationship between the desired (S) and undesired (N) part of the 
waveform. These signals will have a noise (or unwanted content) content higher than desired compared to the desired signal itself. 
\\ The other two defects are considered to be in the same group of audio defects which are the resolution issues: bit depth and 
sample rate discrepancies. \\ Digital audio samples are coded as bit streams, and the resolution or the available voltage levels 
that a sound can have is determined by the bit rate. Signals coded as 16 bit signals could be signals coded in 8 bits but in a 16 
bit container, thus having unused bits. This kind of sounds have a certain tonality and users might be interested in knowing this 
features. The main reason for classifying this defect as such is that memory is wasted in containing no relevant information.\\ 
When an audio signal is recorded at a lower sample rate than the sample rate of its container, some bandwidth remains unused 
because of the nyquist limit. An extensive explanation on effective bandwidth, nyquist limit and bit codification can be found 
on [11]. Some previous work in this can be found on Geoffroy Peeters’ paper [6] where a method to calculate the harmonic cutoff 
frequency which can be extrapolated to the effective bandwidth of a signal.
\item The third objective is to evaluate those algorithms in a subset from Freesound. 
\end{itemize}

\section{Structure of the Report}

The Report will be split into The State of the art, where the algorithms of Essentia will be introduced and explained, the methodology where the developed algorithms will be explained in detail and the results and conclusions, where the algorithms will be evaluated and the outcome of the thesis will be explained.

\newpage



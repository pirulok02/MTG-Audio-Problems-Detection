\normallinespacing

\chapter{Introduction}

\bf User-generated content (UGC) \rm or \bf user-created content (UCC) \rm, refers to content, such as images, 
videos, text and audio, posted by users on online platforms such as social media and wikis. The term "user-generated content" 
entered mainstream usage in the mid-2000s, born in web publishing and new media content production circles. Social media users 
are able to provide key eyewitness content and information that may otherwise have been inaccessible. However, UGC is not solely 
limited to mainstream news or media.\\
UGC is used for a wide range of applications, including problem processing, news, entertainment, advertising, gossip, research 
and sound, which will be the most important application for this thesis. Sound UGC refers to the universal database of sounds 
that are generated by users of a particular service and then uploaded to the service to cater to all the other users. This 
approach to media creation proposes one major caveat which is the amount of data that users can generate in a short period 
of time. If a certain quality of the media is to be had, it is imperative to have a \bf Quality Control(QC) \rm mechanism to 
achieve a quality standard. \\
That QC mechanism can be both human or automated, this project is a first approach to doing an automated QC mechanism for audio 
datasets generated by users. However, instead of removing the media if it does not meet the quality standard, this thesis proposes 
an approach by automatically labeling sounds with the issue that they have in order for the user that searches for the media to 
decide if the defect on the particular piece of media is acceptable or not.\\
As a test set for the algorithms of the thesis, a subset posted in \bf Kaggle \rm for the \bf audio tagging competition \rm 
composed of audio files from \bf Freesound \rm will be used.\\ 
Freesound is a collaborative repository of audio samples started in 2005 at the Music Technology Group (MTG) of Universitat 
Pompeu Fabra. The initial aim of the project was to create an open database of sounds that could also be used for scientific 
research. However Freesound became bigger and bigger and turned into a wide and active community of users which grew well beyond 
any initial goals.\\

\section{Objectives}
The thesis will have three main objectives:
\begin{itemize}
\item The first is to evaluate and tune existing algorithms, which are included in the Essentia library. 
Designed for detecting defects in songs, which its size is much greater than the target audio files of this thesis. 
\item The second one is to create and develop algorithms to detect other issues common in sound files. Three more issues will 
be evaluated in this thesis, low SNR, bit depth discrepancies and sample rate discrepancies. \\ Low SNR (Signal to Noise Ratio) 
signals are defined by the SNR equation, which quantifies the relationship between the desired (S) and undesired (N) part of the 
waveform. These signals will have a noise (or unwanted content) content higher than desired compared to the desired signal itself. 
\\ The other two defects are considered to be in the same group of audio defects which are the resolution issues: bit depth and 
sample rate discrepancies. \\ Digital audio samples are coded as bit streams, and the resolution or the available voltage levels 
that a sound can have is determined by the bit rate. Signals coded as 16 bit signals could be signals coded in 8 bits but in a 16 
bit container, thus having unused bits. This kind of sounds have a certain tonality and users might be interested in knowing this 
features. The main reason for classifying this defect as such is that memory is wasted in containing no relevant information.\\ 
When an audio signal is recorded at a lower sample rate than the sample rate of its container, some bandwidth remains unused 
because of the nyquist limit. An extensive explanation on effective bandwidth, nyquist limit and bit codification can be found 
on [11]. Some previous work in this can be found on Geoffroy Peeters’ paper [6] where a method to calculate the harmonic cutoff 
frequency which can be extrapolated to the effective bandwidth of a signal.
\item The third objective is to evaluate those algorithms in a subset from Freesound. 
\end{itemize}

\section{Structure of the Report}

The Report will be split into The State of the art, where the algorithms of Essentia will be introduced and explained, the methodology
where the developed algorithms will be explained in detail and the results and conclusions, where the algorithms will be evaluated and
the outcome of the thesis will be explained.

\newpage


